\documentclass[letterpaper, 9pt]{article}
\usepackage{amssymb}
\usepackage{mathtools}

\title{Lecture eleven}
\begin{document}
\maketitle
\section{Final - non negative matrix factorization}
Data needs to be vectorized. The W is the code book (basis). There is the code h. We need to develop code books that will describe good features.

From the images we need to generate the code book as well as the code. This method is called a factorization of a vector into two matrices. The code book should extract different features.

In the real-world case we turn many images into a codebook and many codes for each image. Get a factorization of the dataset into a code book and a matirx of codes.

\section{methods}
\begin{itemize}
\item vector quantization
\begin{itemize}
\item clustering - similar to Expectation maximization algorithm.
\item Linde-buzo-gray
\item nearest neighboars
\end{itemize}
With this you can generate basis vectors in the data set. You can then reconstruct the image from the basis vectors. This is one solution to the code book problem.
\item Principle components: [eigen faces]
\begin{itemize}
\item Eigenvectors of the covariance matrix.
\item Eigenfaces: since the eigen vectors are orthoganol you get a new basis from the eigenvectors.
\item You don't need all of the eigenvectors to get good vectors. you want the eigenvectors with the highest eigen value.
\end{itemize}
You can then generate a code book from the eigen vectors to reconstruct the original images.
\item non-negative Matrix factorization
\begin{itemize}
\item basis W and code H are positive or zero
\item error V-V' is minimized
\item Algorithm:
\begin{itemize}
\item Ezpectatioin-maximization (EM)-algorithm
\item W and H are optimized alternatively
\item Error function: distance
\begin{equation}
||V - V'||^2 = \sum_{ij}(v_{ij} - v_{ij}')^2
\end{equation}
Kullback-leiber difference:
\begin{equation}
D(V||V') = \sum_{ij} \left ( v_{ij} log(\frac{v_{ij}}{v_{ij}'}) - v_{ij} + v_{ij}' \right )
\end{equation}
\item  Algorithm:
\begin{equation}
W_{ia} \leftarrow W_{ia} \sum_\mu \frac{V_{i\mu}}{V_{i\mu}'}h_{a\mu}
\end{equation}
\begin{equation}
W_{ia} \leftarrow \frac{W_{ia}}{\sum_j W_{ja}}
\end{equation}
\begin{equation}
h_{a\mu} \leftarrow h_{a\mu} \sum_i W_{ia} \frac{V_{i\mu}}{V_{i\mu}'}
\end{equation}
$i$ is the row and $a$ is the column in $W$. $\mu$ is the column of H and $V/V'$. $\mu$ is running over the training set. $a$ is the row in H.
\end{itemize}
\item Gradient Descent:
\begin{equation}
\frac{D(v||v')}{\partial v'} = - \frac{v}{v'} + 1
\end{equation}
This is a method for converting gradient descent into a multiplication.
\begin{equation}
+ \sum_{ijk} W_{ik}W_{ij} + \sum_{ij}
\end{equation}
missing things.
\end{itemize}
\end{itemize}

The use of code books can reduce the dimensionality of the features in the classification problem. By doing the dimensionality reduction the classifier is given a much better feature to classify.
\section{Literature}
Daniel lee and sebastian Seung: learning the parts of objects by non-negative matrix factorization

Daniel Lee and Sebastian Seung: algorithms for Non-negative Matrix Factorization.

Yuan Wang et al.: fisher Non-negative Matrix Factorization for learning Local Features.

\section{Second approach to building features}
\subsection{Fourier Transform}
A spectrogram is a display of the frequencies in time with respect to a delta t.

\section{Hadamard Transform}
\begin{equation}
H_2 = \frac{1}{\sqrt{2}}
\begin{bmatrix}
1 & 1\\
1 & -1
\end{bmatrix}
\end{equation}

\begin{equation}
\vec{b} = H_2\vec{a}
\end{equation}

\begin{equation}
H_4 = \frac{1}{\sqrt{2}}
\begin{bmatrix}
H_2 & H_2\\
H_2 & -H_2
\end{bmatrix}
\end{equation}

\begin{equation}
b_{2^n} = H_{2^n}a_{2^n}
\end{equation}

\begin{equation}
\begin{bmatrix}
b_1\\
b_2\\
b_3\\
b_4
\end{bmatrix}
 = 
\frac{1}{2}
\begin{bmatrix}
1 & 1 & 1 & 1\\
-1 & 1 & -1 & 1\\
-1 & -1 & 1 & 1\\
1 & -1 & -1 & 1
\end{bmatrix}
\begin{bmatrix}
a_1\\
a_2\\
a_3\\
a_4
\end{bmatrix}
\end{equation}

\begin{equation}
P' = H_{2^n} P H_{2^n}
\end{equation}

\begin{equation}
P = H_{2^n}P'H_{2^n}
\end{equation}

You can perform a low pass filter of an image by removing the high frequency terms.  This is accomplished by setting the high frequency features to zero.

\subsection{FHT}
\begin{equation}
\frac{1}{\sqrt{2}}
\begin{bmatrix}
H_2 & H_2\\
H_2 & -H_2
\end{bmatrix}
\begin{bmatrix}
a_1\\
a_2\\
a_3\\
a_4
\end{bmatrix}
=
\frac{1}{\sqrt{2}}
\begin{bmatrix}
H_2a_u + H_2a_l\\
H_2a|u - H_2a_l
\end{bmatrix}
\end{equation}

\subsection{FT}
Roots of unity: $X^n = 1$. For the case of $n=2$ there are two roots. For the case when $x^4=1$ there are 4 roots. $x_1=1,x_2=-i,x_3=-1,x+4=-i$ are the four roots of unity.

\begin{equation}
F_8 = \frac{1}{\sqrt{n}}
\begin{bmatrix}
\omega_8^0 & \omega_8^0 & \omega_8^0 & \dots & \omega_8^0 \\
\omega_8^0 & \omega_8^1 & \omega_8^2 & \dots & \omega_8^7 \\
\omega_8^0 & \omega_8^2 & \omega_8^4 & \dots & \omega_8^14 \\
\omega_8^0 & \omega_8^3 & \omega_8^6 & \dots & \omega_8^21 \\
\omega_8^0 & \omega_8^4 & \omega_8^8 & \dots & \omega_8^28 \\
\omega_8^0 & \omega_8^5 & \omega_8^10 & \dots & \omega_8^35 \\
\omega_8^0 & \omega_8^6 & \omega_8^12 & \dots & \omega_8^42 \\
\omega_8^0 & \omega_8^7 & \omega_8^14 & \dots & \omega_8^49
\end{bmatrix}
\end{equation}

\begin{equation}
B = FAF
\end{equation}
\begin{equation}
||B|| \leftarrow \text{power spectrum}
\end{equation}


The power spectrum is spatially invariant in the frequency domain. The convolutional property is valid for a torus. Which is what makes the power spectrum spatially invariant, but only when objects do no lie on the edge.

\section{Final}
write a report. The report is going to be about Deep learning.
\begin{enumerate}
\item Read about deep learning.
\item Describe an example of a deep learning network
\item Look at the Activation functions?
\item Look at the topology of the network:
\begin{itemize}
\item pyramids
\item convolution
\end{itemize}
\item Applications
\end{enumerate}

Send PDF, May 7th, Sat Midnight: rojas@inf.fu-berlin.de

Don't write more than 10 pages.
\end{document}